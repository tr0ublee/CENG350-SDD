\documentclass[11pt]{article}
\usepackage[utf8]{inputenc}
\usepackage{hanging}
\usepackage{blindtext}
\usepackage{enumitem}
\usepackage{xcolor}
\usepackage{titlesec}

\usepackage
[       a4paper,
        left=3cm,
        right=3cm
] {geometry}



% =====INTERNET CODE COPIED========================

\titleclass{\subsubsubsection}{straight}[\subsection]

\newcounter{subsubsubsection}[subsubsection]
\renewcommand\thesubsubsubsection{\thesubsubsection.\arabic{subsubsubsection}}
\renewcommand\theparagraph{\thesubsubsubsection.\arabic{paragraph}} % optional; useful if paragraphs are to be numbered

\titleformat{\subsubsubsection}
  {\normalfont\normalsize\bfseries}{\thesubsubsubsection}{1em}{}
\titlespacing*{\subsubsubsection}
{0pt}{3.25ex plus 1ex minus .2ex}{1.5ex plus .2ex}

\makeatletter
\renewcommand\paragraph{\@startsection{paragraph}{5}{\z@}%
  {3.25ex \@plus1ex \@minus.2ex}%
  {-1em}%
  {\normalfont\normalsize\bfseries}}
\renewcommand\subparagraph{\@startsection{subparagraph}{6}{\parindent}%
  {3.25ex \@plus1ex \@minus .2ex}%
  {-1em}%
  {\normalfont\normalsize\bfseries}}
\def\toclevel@subsubsubsection{4}
\def\toclevel@paragraph{5}
\def\toclevel@paragraph{6}
\def\l@subsubsubsection{\@dottedtocline{4}{7em}{4em}}
\def\l@paragraph{\@dottedtocline{5}{10em}{5em}}
\def\l@subparagraph{\@dottedtocline{6}{14em}{6em}}
\makeatother

\setcounter{secnumdepth}{4}
\setcounter{tocdepth}{4}




%========== INTERNET COPIED CODE ENDS===============


%\usepackage{indentfirst}
\usepackage{comment}
\usepackage{natbib}
\usepackage{graphicx}
\usepackage{float}
\usepackage{color}   %May be necessary if you want to color links
\usepackage{hyperref}
\setcounter{secnumdepth}{4}
\hypersetup{
    colorlinks,
    citecolor=black,
    filecolor=black,
    linkcolor=black,
    urlcolor=black
}

\renewcommand{\contentsname}{Table of Contents}

\begin{document}


\begin{titlepage}

\newcommand{\HRule}{\rule{\linewidth}{0.5mm}} % Defines a new command for the horizontal lines, change thickness here

%----------------------------------------------------------------------------------------
%	LOGO SECTION
%----------------------------------------------------------------------------------------


%----------------------------------------------------------------------------------------

\center % Center everything on the page

%----------------------------------------------------------------------------------------
%	HEADING SECTIONS
%----------------------------------------------------------------------------------------
\textsc{}\\[3cm] % Name of your university/college
\textsc{\Huge \bfseries Software Design Description}\\[1cm] % Name of your university/college


%----------------------------------------------------------------------------------------
%	TITLE SECTION
%----------------------------------------------------------------------------------------
\makeatletter
\HRule \\[0.4cm]
{ \LARGE  Amazon Go }\\[0.1cm] % Title of your document
\HRule \\[4cm]
 
%----------------------------------------------------------------------------------------
%	AUTHOR SECTION
%----------------------------------------------------------------------------------------

\begin{minipage}{0.4\textwidth}
\begin{flushleft} \large
\emph{Authors}\\
\@author % Your name
\end{flushleft}
\end{minipage}
~
\begin{minipage}{0.4\textwidth}
\begin{flushright} \large
\emph{Student 1:} \\
Alperen ÇAYKUŞ \\[0.5em] % Supervisor's Name
2237170 \\[1.2em] % Supervisor's Name

\emph{Student 2:} \\
Aytaç Anıl Durmaz \\[0.5em] % Supervisor's Name
2237295 \\[1.2em] % Supervisor's Name
\end{flushright}
\end{minipage}\\[2cm]
\makeatother

% If you don't want a supervisor, uncomment the two lines below and remove the section above
%\Large \emph{Author:}\\
%John \textsc{Smith}\\[3cm] % Your name

%----------------------------------------------------------------------------------------
%	DATE SECTION
%----------------------------------------------------------------------------------------

%{\large \today}\\[2cm] % Date, change the \today to a set date if you want to be precise

\vfill % Fill the rest of the page with whitespace

\end{titlepage}


\newpage

\begin{center}
    \Huge{Change History} 
\end{center}{}
 


\begin{table}[H]
\begin{center}
    

\begin{tabular}{|l|l|}
\hline
\textbf{\Huge{Version}} & \textbf{\Huge{Date}}  \\ \hline
 %
\end{tabular}
\end{center}
\end{table}

\newpage



\begin{flushleft}
    \tableofcontents
\end{flushleft}

\newpage

\begin{flushleft}
    \listoffigures
\end{flushleft}

\newpage

\begin{flushleft}
    \listoftables
\end{flushleft}

\newpage


\section{Introduction}

    \subsection{Purpose of the System}

    % Copied from the SRS. Ask if this is valid.
    The purpose of this system, namely 'Amazon Go', is to enable customers to shop without making them wait in line with a few
    requirements only, which are a smart phone and an Amazon account with a registered credit card. To achieve this purpose, Amazon has opened lots of stores, 
    which are highly equipped with cutting edge sensors and cameras. These cameras and sensors track the customer and products,
    and after the customer is done with shopping, she/he just walks out of store. The amount of his/her shopping will be automatically 
    deduced from his/her specified payment method.
        

    \subsection{Scope}
        \begin{itemize}
        \item System will have a mobile application, which will enable users (i.e, store customers) to interact with the system. 
            Customers will log in to store by using the QR code provided to them by their mobile application. 
            Mobile application is also the place where customers sign up to the system, see their current cart status, and past shoppings.
        \item System will use remote servers to keep data of the customers, such as current shopping session, payment information, etc. 
            Remote server will also communicate with the mobile application to enable a customer to see his/her current shopping session, 
            past shoppings, etc. 
        \item System will have physical stores, which are equipped with very sensitive sensors and cameras. Using Artificial Intelligence 
            and Computer Vision algorithms, these cameras and sensors will gather information from the store and communicate with the remote server.
        
        \item System will use several APIs to communicate with the specified payment method of the customers, such as bank accounts. 
            By doing so, system will be able to withdraw money from the customer's account when she/he is done with shopping.
        
        \item System will use a database to store customer related information (such as user ID), temporary customer shopping cart, store workers' information,
            products' information and a database table for the system admins. Those mentioned tables such as customer related information 
            actually consist of multiple tables. Only the system admins are able to make changes and read the database.
        
        \item System will keep log of all the shoppings of the users on the remote server for legal purposes. Only system admins can see these logs.
        
        \item System will also have a store worker interface to enable the store worker to troubleshoot in case something goes wrong in the store, 
        such as sensor malfunctioning, incorrect amount of money withdrawn from the customer, wrong product was added to a customer's cart, etc.
        \end{itemize}



    \subsection{Stakeholders and their concerns}
        \begin{itemize}
            
            \item {\textbf{Amazon:}}
                Being the owner of the whole system and the store, Amazon's main concern is reliability of the system so that Amazon will 
                not lose money due to sensor failures or due to customers tricking the sensors and being charged less than they are supposed to be charged.
                Privacy of the customers is also important as this may have legal consequences.

            \item {\textbf{Users:}}
                Users are actually the customers that visit Amazon GO stores. They have four main concerns, which are, firstly, not waiting 
                in line to pay for what they had bought; secondly, precision of the sensors so that they do not pay extra money
                due to sensors' fault; thirdly, security of their personal information such as their credit card information, place of residence etc., 
                which are stored in the Amazon servers; and lastly, having a mobile application which is easy to use.
               
            \item {\textbf{System Developers:}}
                Developers are very critical for this system, because it completely relies on the software. Their main concern is maintainability, due
                to complexity of the system. For this purpose, the documentation and the system itself should be organized very well. Also, the system 
                must be set up in a way that if one component fails, restoration of this component and integration of the new component, which replaces the 
                faulty one, with the system must be easy.

            \item {\textbf{IT Staff:}}
                IT Staff are basically the system administrators, who are responsible to maintain databases and communicate with the store workers when needed.
                They are essential in the continuity of the system. Their major concern is that sustainability of the database, which is created by the system
                developers when the system is being developed. 

            \item {\textbf{Store Workers:}}
                Store workers are the people that are responsible for the product placement and shelf maintenance. They are mostly concerned about the 
                illegal actions. Therefore, they should be trained to what to do for the most of the possible scenarios. 
                Their main concerns are being able to easily reach a system admin in case of a failure, and being able to contact with the security
                in case of an illegal action by the customers.  
        \end{itemize}

\section{References}
%% @@@@@@@@@@@@@@@@@@@@@@@@@@@@@@@@@@@@ OLD REFERENCES !!!!!
\textbf{This document is written with respect to the specifications of the document below:\\}

1016-2009 - IEEE Standard for Information Technology--Systems Design--Software Design Descriptions
$$ $$ 
\setlength{\parindent}{1 cm}
\textbf{Other sources:\\}
\begin{hangparas}{1cm}{1}
 \par
 Cheng, A. (2019, January 13). Why Amazon Go May Soon Change The Way We Shop. Retrieved from \textit{https://www.forbes.com/sites/andriacheng/2019/01/13/why-amazon-go-may-soon-change-the-way-we-want-to-shop} 


 \par
 Wingfield, N. (2018, January 21). Inside Amazon Go, a Store of the Future.\\ 
 Retrieved from \textit{https://www.nytimes.com/2018/01/21/technology/inside-amazon-go-a-store-of-the-future.html }
   
\end{hangparas}

\section{Glossary}
\begin{table}[H]
    \begin{center}
    \begin{tabular}{|p{5cm}|p{8cm}|}
    \hline
    \textbf{Term} & \textbf{Definition}   \\ \hline
    User & A customer that is signed up to Amazon Go application and has a unique user ID. \\ \hline
    
    Database & A MySQL database.  \\ \hline
    User ID & A unique number which can be used to identify a user.  \\ \hline
    Store Worker & A real person working in an Amazon Go store.  \\ \hline
    API & Application Programming Interface   \\ \hline
    %HTTPS & Hypertext Transfer Protocol Secure  \\ \hline
    %TCP & Transmission Control Protocol \\ \hline
    Mobile Application & An application that must be installed to an Android or iOS device in order to access Amazon GO features.\\ \hline
    QR Code & An image which contains information about the user. It is generated by the mobile application.  \\ \hline
    System admins  & The people who are responsible for the maintenance and the order of the Amazon GO system.   \\ \hline
    Remote server & A real physical computer away from the stores, at the Amazon HQ, which can be accessed via SSH by system admins.   \\ \hline
    SSH & Secure Shell   \\ \hline
    Sensor & A physical electronic device that collects information about the environment it is currently in.   \\ \hline
    Product & The goods that are sold at the Amazon GO stores.   \\ \hline
    % &   \\ \hline

    \end{tabular}
    \caption{Glossary}
    \label{glossary}
      \end{center}
\end{table}

\newpage

\section{Architectural Views}
    \subsection{Context View}

    In this viewpoint, the systems that Amazon GO interacts with are shown on the \nameref{contextDiagram} below. The following \nameref{ucaseDiagram} 
    shows the actors and their possible interactions with different scenarios. The detailed information about these use case functions can be found
    on the tables after the Use Case Diagram. These tables give detailed information for each use case function including alternative scenarios.
    During the implementation, these tables shall be considered and implementation must follow the mentioned scenarios.

    \begin{center}
        \begin{figure}[H]
            \includegraphics[width=\linewidth]{Images/contextDiagram.png}
            \caption{Context Diagram}
            \label{contextDiagram}
        \end{figure}
        \end{center}

    \begin{center}
        \begin{figure}[H]
            \includegraphics[width=\linewidth]{Images/UseCaseDiagram.png}
            \caption{Use Case Diagram}
            \label{ucaseDiagram}
        \end{figure}
        \end{center}

    \begin{table}[H]
        \begin{centering}
        \begin{tabular}{|p{2.5cm}|p{9cm}|}
        \hline
        Use case name & Customer recognition \\ \hline
        Actors        & Customers, Sensors \\ \hline
        Description   & Immediately after the customer gets his/her QR code scanned, environmental cameras must track his/her movements and positions continuously. Cameras must regularly recognize the customers with their face, clothes etc. and must communicate with the remote server until the customer leaves the store. In order to not to lose visual contact with the customer, every angle in the store must be watched by the cameras and the sensors. \\ \hline
        Data          & Customers' physical details such as clothes, face, and body movements.  \\ \hline
        Stimulus      & The customer getting his/her QR code scanned. \\ \hline
        Response      & Match the user ID with the acquired visual data of the customer. \\ \hline
        Basic Flow    & 
        1- Upon QR scanning, visual sensors receive a user ID and identify the unknown customer with the ID. \newline
        2- Visual sensors track customer's movements and position continuously. \newline
        3- Customer picks up a product. \newline
        4- Customer information is sent to the server.\newline
        5- Flow goes back to step 2. \\ \hline
        Alternative
            Flow \#1  & 
        3- Customer puts a product back to a shelf. \newline
        4- Customer information is sent to the server. \newline
        5- Flow goes back to step 2. \\ \hline
        Alternative
            Flow \#2  &
        3- Customer leaves the store. \newline
        4- Customer information is sent to the server. \newline
        5- System stops tracking the customer. \\ \hline
        Comments      & Immediately after a customer logs in to store, sensors must assign a unique customer ID to that user. \\ \hline
        
        \end{tabular}
        \caption{Customer Recognition}
        \label{tab3}
        \end{centering}
        \end{table}    
        
        
        
        \begin{table}[H]
        \begin{centering}
        \begin{tabular}{|p{2.5cm}|p{9cm}|}
        \hline
        Use case name & Product recognition  \\ \hline
        Actors        & Sensors \\ \hline
        Description   & Right after a product is placed to a shelf by the store staff and the product is marked as 'exist' on the store database, the corresponding ID which belongs to the same product set must be assigned to the product. After that point, until someone buys the product and leaves the store, product must be tracked by using its weight information, visual information, and shelf location. A customer is allowed to take a product, put it into his/her bag, and put the product back to any shelf. In that case, for some time, the visual contact with the product would be lost, but after the product is put back to the shelf, system must recover and continue tracking this product. A customer can put a product back to a different shelf, in that case, the product must be recognized by its visual data and weight again. \\ \hline
        Data          & Weight, physical characteristics, and the location of products. \\ \hline
        Stimulus      & A product is placed to a shelf by the store worker and marked as exist in the store. \\ \hline
        Response      & Match the product with the products in the database. \\ \hline
        Basic Flow    & 
        1- A product is placed to a shelf by the staff. \newline
        2- Query the database with the shelf location, visual data and weight. \newline
        3- Assign the received product ID to corresponding product set.
        \newline
        4- Sensors work continuously to detect product movements. \newline
        5- A product is taken from a shelf. \newline
        6- Product information is sent to the server. \newline
        7- Flow goes back to step 4. \\ \hline
        Alternative
            Flow      & 
        5- A product is placed to a shelf by a customer. \newline
        6- Product information is sent to the server. \newline
        7- Flow goes back to step 4. \\ \hline
        Comments      & Initially, products must be registered to the database and must be assigned to a shelf. \\ \hline
        
        \end{tabular}
        \caption{Product Recognition}
        \label{tab7}
        \end{centering}
        \end{table}    
        

        \begin{table}[H]
        \begin{centering}
        \begin{tabular}{|p{2.5cm}|p{9cm}|}
        \hline
        Use case name & Scan login QR code \\ \hline
        Actors        & Customers \\ \hline
        Description   & A procedure to log users in to store by scanning their QR code via turnstiles. Turnstiles must allow access after a successful scan. Turnstiles must not allow the customers to enter the store if the scan fails and must display the reason. For security reasons, if the same QR is scanned while there is a shopping session with that QR code, login must not be allowed and QR code must be disabled for security purposes. Additionally, customer must be informed about every login with an SMS.    \\  \hline
        Data          & User information. \\ \hline
        Stimulus      & Users showing their QR code to turnstiles. \\ \hline
        Response      & Validation or invalidation of the QR code.  \\ \hline
        Basic Flow    &
        1- Customer shows the QR code to turnstile QR scanner. \newline
        2- Turnstile scans the QR and sends the data to the server. \newline
        3- Server processes the store information and the customer information. \newline
        4- Server responds with success. \newline
        5- Turnstile gives access to the customer. \newline
        6- Server sends SMS notification to the customer. \\ \hline
        Alternative
            Flow      & 
        4- Server responses with failure. \newline
        5- Failure reason is displayed. \newline
        6- Server sends SMS notification to the customer. \\ \hline
        Comments      & Customer must be a registered member. GPS can be used to improve the security.\\ \hline
        
        \end{tabular}
        \caption{Scan Login QR Code}
        \label{tab}
        \end{centering}
        \end{table}    
        
        
                
        \begin{table}[H]
        \begin{centering} 
        
        \begin{tabular}{|p{2.5cm}|p{9cm}|}
        \hline
        Use case name & Pick Product  \\ \hline
        Actors        & Customers, Sensors  \\ \hline
        Description   & A procedure to detect customers when they pick up a product from the shelves. When a customer picks a product, weight sensors must detect the change and must send the related data, also visual sensors must recognize the customer who picks up the product and provide additional data about the product. If the customer picks up multiple products at the same time, sensors must be precise enough to handle the situation.  \\ \hline
        Data          & Customer information, product's price and other information
                        related to that product. \\ \hline
        Stimulus      & Customers picking products. \\ \hline
        Response      & Add the picked product to customer's cart, update the cart information. \\ \hline
        Basic Flow    & 
        1- Customer picks a product up from a shelf. \newline
        2- Weight sensors detect the change. \newline
        3- Visual sensors detect the customer action and assists product recognition. \newline
        4- Sensors send data to server. \newline
        5- Server processes data, adds the related product to customer's cart.
        \\ \hline
        Alternative
            Flow      & - \\ \hline
        Comments      & Customers must get their QR code scanned successfully before shopping. \\ \hline
        
        \end{tabular}
        \caption{Pick Product}
        \label{tab1}
        \end{centering}
        \end{table}    
        
        
        
        \begin{table}[H]
        \begin{centering}
        \begin{tabular}{|p{2.5cm}|p{9cm}|}
        \hline
        
        % uml
        Use case name & Put back product  \\ \hline
        Actors        & Customers, Sensors \\ \hline
        Description   & A procedure to detect customers when they put a product back to shelves. When a customer puts a product back to a shelf, weight sensors must detect the change and must send shelf information, weight information and other related data, also visual sensors must recognize the customer who puts the product back and must provide additional data about the product. System must analyze the cart to be precise about the product. If the customer puts back multiple products at the same time, sensors must be precise enough to handle the situation.\\ \hline
        Data          & Customer information, product's price and other information.  \\ \hline
        Stimulus      & Customers putting a product back to shelf. \\ \hline
        Response      & Delete the related product from the customer's cart. \\ \hline
        Basic Flow    & 
        1- Customer puts a product back to a shelf. \newline
        2- Weight sensors detect the change. \newline
        3- Visual sensors detect the customer action and assists product recognition. \newline
        4- Sensors send data to server. \newline
        5- Server processes data, removes the related product from customer's cart. \\ \hline
        Alternative
            Flow      & - \\ \hline
        Comments      & The product must have been picked up beforehand and this must be in the same session. \\ \hline
        
        \end{tabular}
        \caption{Put Back Product}
        \label{tab2}
        
        \end{centering}
        \end{table}    
        
        
                
        \begin{table}[H]
        \begin{centering}
        \begin{tabular}{|p{2.5cm}|p{9cm}|}
        \hline
        Use case name & View cart info  \\ \hline
        Actors        & Customers, Mobile Application \\ \hline
        Description   & A customer must be able to see it's current cart status 
                        by using the mobile app. Customer shall see the products she/he picked up, with their quantity and price. Total price must be also displayed. \\ \hline
        Data          & The customer's cart status. \\ \hline
        Stimulus      & The customer pressing the 'My Cart' button on the app.  \\ \hline
        Response      & Show the customer's current cart status.  \\ \hline
        Basic Flow    & 
        1- Customer opens the app. \newline
        2- Customer logins if not signed in already. \newline
        3- Application opens the cart screen automatically. \newline
        4- Application sends view cart request to the server. \newline
        5- Server responds with the related data. \newline
        6- Application shows the cart status. \\ \hline
        Alternative
            Flow      & - \\ \hline
        Comments      & Data must be fetched from the remote server in case of an attempt to cheat. User must be shopping to be able to see the cart screen.\\ \hline
        
        \end{tabular}
        \caption{View Cart Info.}
        \label{tab4}
        \end{centering}
        \end{table}    
        
        
        
        \begin{table}[H]
        \begin{centering}        
        \begin{tabular}{|p{2.5cm}|p{9cm}|}
        \hline
        Use case name & Enter personal and payment information  \\ \hline
        Actors        & Customers, Mobile Application  \\ \hline
        Description   & When a customer signs up, s/he must provide personal 
                        information and payment method. If the payment method is invalid, application must request a valid method and must not allow another action to be taken by the customer until a valid payment method is provided.  \\ \hline
        Data          & Personal data and payment data.  \\ \hline
        Stimulus      & When a customer signs up or updates his/her information. \\ \hline
        Response      & Validation or invalidation of the payment method. If validated, sign up or update the user. \\ \hline
        Basic Flow    & 
        1- Application shows a personal information form or a payment form. \newline
        2- User enters information. \newline
        3- Application checks the validity of the information at the same time. \newline
        4- If everything is valid, a button is enabled to finish the process. \newline
        5- User presses the button. \newline
        6- Server inserts the related information to the database. \\ \hline
        Alternative
            Flow      & - \\ \hline
        Comments      & If the customer is a member, then instead of signing him/her up, update the information. \\ \hline
        
        \end{tabular}
        \caption{Enter Personal and Payment Information}
        \label{tab5}
        \end{centering}
        \end{table}    
        
        
        
        \begin{table}[H]
        \begin{centering}        
        \begin{tabular}{|p{2.5cm}|p{9cm}|}
        \hline
        Use case name & End shopping session  \\ \hline
        Actors        & Mobile Application, Sensors \\ \hline
        Description   & Visual sensors must detect when the customer leaves the store and report to remote server. Details about the shopping session must be visible from the application after session ends.  \\ \hline
        Data          & Customer ID, store information.  \\ \hline
        Stimulus      & Customer leaving the store. \\ \hline
        Response      & Session information. \\ \hline
        Basic Flow    & 
        1- Customer walks out from the store. \newline
        2- Visual sensors detects the action. \newline
        3- End shopping session request is sent to servers with the customer ID. \newline
        4- Server processes the request and saves the sessions information to the database. \newline
        5- Visual sensors stop tracking that customer. \\ \hline
        Alternative
            Flow      & - \\ \hline
        Comments      & Log the current session to database for possible further use.  \\ \hline
        
        \end{tabular}
        \caption{End Shopping Session}
        \label{tab6}
        \end{centering}
        \end{table}    
        
        
                
        \begin{table}[H]
        \begin{centering}
        \begin{tabular}{|p{2.5cm}|p{9cm}|}
        \hline
        Use case name & Generate login QR code \\ \hline
        Actors        & Mobile Application \\ \hline
        Description   & Application must generate a unique QR code to enable customers log in. A generated QR code can be used multiple times until it is disabled due to security or user prompt. QR codes must be user specific. \\ \hline
        Data          & Customer information. \\ \hline
        Stimulus      & User pressing the 'QR Code' button on the mobile app. \\ \hline
        Response      & Produce and show the QR code \\ \hline
        Basic Flow    & 
        1- A customer signs up. \newline
        2- Application requests a QR code for the customer. \newline
        3- Server generates a QR code that is unique to the customer.\\ \hline
        Alternative
            Flow      &
        1- Customer presses the Refresh QR button. \newline
        2- Application requests a QR code for the customer. \newline
        3- Server generates a QR code that is unique to the customer.\\ \hline
        Comments      & QR code must be generated and sent by the remote server in case of a cheating attempt. \\ \hline
        
        \end{tabular}
        \caption{Generate Login QR Code}
        \label{tab8}
        \end{centering}
        \end{table}    
        
        
                
        \begin{table}[H]
        \begin{centering}
        \begin{tabular}{|p{2.5cm}|p{9cm}|}
        \hline
        Use case name & Withdraw money \\ \hline
        Actors        & Mobile Application \\ \hline
        Description   & When user leaves the store, withdraw money from his/her registered payment method. System must use the payment method provided by the customer. If the payment attempt fails due to lack of money or other reasons, notifications must be sent regularly until customer pays. Customer must not be allowed to shop until she/he pays.    \\ \hline
        Data          & User ID. \\ \hline
        Stimulus      & Upon the 'End shopping session' procedure. \\ \hline
        Response      & Confirmation that payment is successful.  \\ \hline
        Basic Flow    & 
        1- System is informed by the End Shopping Session function. \newline
        2- System requests a payment from the user's payment method via the Payment API. \newline
        3- Payment is successful. \newline
        4- Receipt is created.  \\ \hline
        Alternative
            Flow      & 
        3- Payment is failed. \newline
        4- An SMS notification is sent to the customer. \newline
        5- The customer is blacklisted until the she/he pays. \\ \hline
        Comments      & Further actions can be taken by the authorities if a payment is not received for a long time. \\ \hline
        
        \end{tabular}
        \caption{Withdraw Money}
        \label{tab9}
        \end{centering}
        \end{table}    
        
        
                
        \begin{table}[H]
        \begin{centering}
        \begin{tabular}{|p{2.5cm}|p{9cm}|}
        \hline
        Use case name & View store stock  \\ \hline
        Actors        & System Administrator \\ \hline
        Description   & System admins must be able to see current stock status of each store and main warehouse. User must be warned about low stocks.\\ \hline
        Data          & Store ID, admin ID. \\ \hline
        Stimulus      & User pressing the 'View Stock' button on the mobile app. \\ \hline
        Response      & Stock information .\\ \hline
        Basic Flow    &
        1- System admin opens the administration page. \newline
        2- User presses the stock status tab. \newline
        3- Overall stock status of each store is listed to the user. Stocks are colored according to the amount. \newline
        4- User selects a store. \newline
        5- Detailed stock status of the selected store is shown to the user.
        \\ \hline
        Alternative
            Flow      & - \\ \hline
        Comments      & User must be authorized to view stock information. \\ \hline
        \end{tabular}
        \caption{ View Store Stock }
        \label{tab_10}
        \end{centering}
        \end{table}    
        
        
                
        \begin{table}[H]
        \begin{centering}
        \begin{tabular}{|p{2.5cm}|p{9cm}|}
        \hline
        Use case name & View system logs  \\ \hline
        Actors        & System Administrators \\ \hline
        Description   & Administrators must be able to see previous shopping session details of customers. Any inconsistency in sessions must be reported. \\ \hline
        Data          & Time and date, Store ID, Customer ID \\ \hline
        Stimulus      & Administrators reaching the log files. \\ \hline
        Response      & Previous cart information, payment method  \\ \hline
        Basic Flow    & 
        1- System admin opens the administration page. \newline
        2- User presses the view logs tab. \newline
        3- Several filter options are shown to the user. \newline
        4- User fills some of the filtering fields. \newline
        5- User presses the search button. \newline
        6- Results are listed. \\ \hline
        Alternative
            Flow      & - \\ \hline
        Comments      & Only administrators can see the log files. \\ \hline
        \end{tabular}
        \caption{View System Logs}
        \label{tab_11}
        \end{centering}
        \end{table}    

        %% == NEW USES CASES == %%
        \begin{table}[H]
            \begin{centering}
            \begin{tabular}{|p{2.5cm}|p{9cm}|}
            \hline
            Use case name & Mark Product as Favorite \\ \hline
            Actors        & Customers, Mobile Application \\ \hline
            Description   & Customers should be able to add a product that they buy frequently or a product that they wish to buy to their favorite products list. \\ \hline
            Data          & Product ID, Customer ID \\ \hline
            Stimulus      & A user pressing the 'Star' on the page of the product using the mobile application. \\ \hline
            Response      & A message indicating whether the operation was successful or not.  \\ \hline
            Basic Flow    & 
            1- Customer logs in to the mobile application. \newline
            2- User navigates to the product's page on the mobile application via the search button or via the categories. \newline
            3- System loads the product page. \newline
            4- User presses the dimmed 'Star' icon on the loaded page. \newline
            5- 'Star' icon is lighted up and the product is added to the user's favorites. \\ \hline
            Alternative
                Flow      & 
                4- User presses the already lighted up 'Star' icon \newline
                5- 'Star' icon is dimmed and the product is removed from the user's favorites. \\ \hline
                
            Comments      & On each press to 'Star' icon, the icon is basically toggled. \\ \hline
        \end{tabular}
        \caption{Mark Product as Favorite}
        \label{tab_12}
        \end{centering}
        \end{table}    

        \begin{table}[H]
            \begin{centering}
            \begin{tabular}{|p{2.5cm}|p{9cm}|}
            \hline
            Use case name & Report an Issue  \\ \hline
            Actors        & Customers, Mobile Application \\ \hline
            Description   & A user should be able to communicate with the system admins in case of any problems or questions.\\ \hline
            Data          & User ID, User Message \\ \hline
            Stimulus      & User presses the 'Report an Issue' button at the main page of the mobile application. \\ \hline
            Response      & Validation that issue is reported or not.  \\ \hline
            Basic Flow    & 
            1- Customer logs in to the mobile application. \newline
            2- User navigates to the issue reporting page by pressing the 'Report an Issue' button. \newline
            3- A textbox appears on the screen which waits for user to enter his message.\newline
            4- User enters his/her message and presses the send button. \newline
            5- A notification appears if the message is transmitted successfully or not. \\ \hline
            Alternative
                Flow      & - \\ \hline
            Comments      & If a user sends two seperate messages consecutively, these messages shall be merged to avoid spam. Also a user 
                            should be able to send only one message in ten minutes. \\ \hline
        \end{tabular}
        \caption{Report an Issue}
        \label{tab_13}
        \end{centering}
        \end{table}    


        \begin{table}[H]
            \begin{centering}
            \begin{tabular}{|p{2.5cm}|p{9cm}|}
            \hline
            Use case name & Process the Issue  \\ \hline
            Actors        & System Administrators \\ \hline
            Description   & System admins inspect the issue provided by the user and take action accordingly. \\ \hline
            Data          & Response Message, User ID \\ \hline
            Stimulus      & An admin pressing the 'Inspect Issues' button on the admin page and selecting an issue.\\ \hline
            Response      & The status of the selected issue is updated (such as ongoing, solved etc.) \\ \hline
            Basic Flow    & 
            1- An admin logs into the admin page using his Admin ID and password. \newline
            2- The admin presses the 'Inspect Issues' button on the admin page.\newline
            3- A list of unsolved issues are shown to the admin.  \newline
            4- The admin selects an issue by the date (early messages first). \newline
            5- The admin takes action (such as inspecting the logs) and sends a reply to the issuer. \\ \hline
            Alternative
                Flow      & 
            6- The admin marks the issue as solved.\\ \hline
            Comments      & \\ \hline
        \end{tabular}
        \caption{Process the Issue}
        \label{tab_14}
        \end{centering}
        \end{table}    



    \subsection{Composition View}
        In this viewpoint, the components of the system are shown from a top-level point of view. Also, the design rationale for 
        each decision is provided right after the \nameref{componentDia}. % \nameref here
        \begin{center}
            \begin{figure}[H]
                \includegraphics[width=\linewidth]{Images/ComponentDiagram.png}
                \caption{Component Diagram}
                \label{componentDia}
            \end{figure}
        \end{center}

     
    \subsection{Information View}
    \subsection{Interface View}






\end{document}




% ========== MIGHT BE NECESSARY==================%

%``I always thought something was fundamentally wrong with the universe'' \citep{adams1995hitchhiker}
%\bibliographystyle{plain}
%\bibliography{references}
